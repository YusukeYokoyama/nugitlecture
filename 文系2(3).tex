\documentclass[a4j, 11pt]{jarticle}
\usepackage{amsmath}
\usepackage{amssymb}
\usepackage{amsthm}

\begin{document}

\subsection*{(3)}
 (2) より P$_n$, \ P$_{n + 1}$ の座標はそれぞれ $(\cos n \alpha, \sin n \alpha), \ (\cos (n + 1) \alpha, \sin (n + 1) \alpha)$ であるので,
  \begin{align*}
   \triangle \textrm{P}_n \textrm{O} \textrm{P}_{n + 1} &= |\textrm{O} \textrm{P}_n| |\textrm{O} \textrm{P}_{n + 1}| \frac{1}{2} \sin \angle \textrm{P}_n \textrm{O} \textrm{P}_{n + 1} \\
                                   &= \frac{1}{2} \left| \sin (n + 1) \alpha \cos n \alpha - \sin n \alpha \cos (n + 1) \alpha \right| \\
                                   &= \frac{1}{2} \left| \sin \alpha \right| \\
                                   &= \frac{1}{2} \sin \alpha \\
  \end{align*}
 となる. ここで $\displaystyle \tan \left( \frac{\alpha}{2} \right) = k$ より, $\sin \alpha = \displaystyle \frac{2k}{1 + k^2}$ なので, 
  \begin{align*}
   \triangle P_n O P_{n + 1} &= \frac{1}{2} \sin \alpha = \frac{k}{1 + k^2}
  \end{align*}
 となる.

\end{document}
