\documentclass{jsarticle}
\usepackage{amsmath}

\begin{document}
\section{平成31年度名古屋大学文系大問2(1)の解答}
$n=0$のとき,$\text{P}_0$の座標が$(1,0)$なので
\[
\begin{cases}
x_1&=x_0-k(y_0+y_1) \\
y_1&=y_0+k(x_0+x_1)
\end{cases}
\text{すなわち}
\begin{cases}
x_1&=1-ky_1 \\
y_1&=k(1+x_1)
\end{cases}
\]
である.これより
\[
\begin{cases}
ky_1=1-x_1 \\
ky_1=k^2(1+x_1)
\end{cases}
\]
であるから
\[
1-x_1=k^2(1+x_1)
\text{ゆえに}
x_1=\frac{1-k^2}{1+k^2}
\]
が導かれる.さらに
\[
ky_1=1-\frac{1-k^2}{1+k^2}=\frac{2k^2}{1+k^2}
\text{ゆえに}
y_1=\frac{2k}{1+k^2}
\]
である.
\end{document}